\documentclass[12pt]{article}
\usepackage{amsmath, amsthm, amsfonts,amssymb}
\usepackage[utf8]{inputenc}
\usepackage[spanish]{babel}
\usepackage{multicol}
\usepackage{listings}
\lstset{basicstyle=\footnotesize\ttfamily,breaklines=true}
\usepackage{alltt}
\usepackage{graphicx}
\usepackage{hyperref}


\usepackage{geometry}
\geometry{left=2.5cm, right=2.5cm, top=2cm, bottom=3cm}

\begin{document}
% Cambia el nombre cuadro por tabla
\renewcommand{\listtablename}{Indice de tablas}
\renewcommand{\tablename}{Tabla}

\begin{figure}
\begin{minipage}{2.5cm}
\includegraphics[width=0.8\textwidth]{LogoUC-BN.pdf}
\end{minipage}
\begin{minipage}{14.5cm}
\vspace{4mm}
{\sc PONTIFICIA UNIVERSIDAD CAT\'OLICA DE CHILE}\\
Escuela de Ingeniería\\
Departamento de Ingeniería Eléctrica\\
Nombre curso \\
2017-1\\
\vspace{0mm}
\hrulefill
\end{minipage}
\end{figure}
\phantom{""}
\vspace{-11mm}
\begin{flushleft}
\scriptsize
\textbf{Profesores:}  Felipe Nuñez \\
\textbf{Ayudante:}
\end{flushleft}
\vspace{-4mm}
\begin{center}
\Large{\textbf{Uso de Photons}}
\end{center}
\normalsize


\section{Instalación}
Esta sección se subdivide en tres, instalación general y  dos modos que se usarán con frecuencia; azul (\textit{listening mode}) y amarillo (DFU). 
\subsection{General}

Instalar Drivers
Instalar node.js
Links de descarga aqui:

\medskip

\hyperref[https://docs.particle.io/guide/getting-started/connect/photon/
]{https://docs.particle.io/guide/getting-started/connect/photon/
}


\medskip
Una vez instalados en cmd hacer
\medskip
\begin{verbatim}
npm install -g particle-cli
\end{verbatim}

\subsection{Azul \textit{Listening mode}}

Es posible que la instalación del driver no funcione, (ver administrador de dispositivos en caso de que exista una alerta al poner el photon en \textit{listening mode}) en cuyo caso ir al siguiente link:

\medskip

\hyperref[https://community.particle.io/t/installing-the-usb-driver-on-windows-serial-debugging/882
]{https://community.particle.io/t/installing-the-usb-driver-on-windows-serial-debugging/882
}




\subsection{Amarillo \textit{DFU}}

Este tutorial explica bien como instalar los drivers para utilizar DFU-utils. Utilizaremos esto para subir código sin tener que utilizar la ''nube''.

\medskip

\hyperref[https://community.particle.io/t/tutorial-installing-dfu-driver-on-windows-24-feb-2015/3518
]{https://community.particle.io/t/tutorial-installing-dfu-driver-on-windows-24-feb-2015/3518
}



\section{Setup}

Si los drivers y programas están bien instalados, no deberían aparecer errores en los siguientes pasos.
\medskip
Poner el photon en \textit{DFU mode} para realizar una actualización de firmware y escribir en cmd: 
\medskip
\begin{verbatim}
particle update
\end{verbatim}
\medskip
Luego nos conectamos a wifi, poner el photon en \textit{listening mode} (azul) y escribir en consola:
\medskip
\begin{verbatim}
particle setup
\end{verbatim}
\medskip
y seguir todas las opciones default (e.g. (y/N) elegir n)

\section{Blink led}

En la carpeta de codigos de ejemplo se ven archivos arduino (.ino) y un .bin.

\medskip

Particle puede compilar y flashear archivos .ino, para eso poner el photon en modo DFU (amarillo) y realizar lo siguiente en consola:

\medskip

\begin{verbatim}

particle compile photon Codigos_Ejemplo/blink/blink.ino --saveTo software.bin
\end{verbatim}
\medskip
El código anterior pasará el codigo fuente a lenguaje de máquina. Luego necesitamos subir el programa al photon:
\medskip
\begin{verbatim}
particle flash --usb software.bin
\end{verbatim}
\medskip
El photon debería blinkear (D7). Notar que ''respira blanco'', esto quiere decir que el módulo wifi está apagado. En la carpeta sockk hay unos códigos en donde el photon se conecta a la red, leer README.
\end{document}